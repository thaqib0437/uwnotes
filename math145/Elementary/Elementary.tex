\documentclass[16pt,a4paper]{article}
\usepackage[utf8]{inputenc}
\usepackage{amsmath}
\usepackage{amsfonts}
\usepackage{amssymb}
\usepackage{amsthm}
\usepackage{physics}



\usepackage[many]{tcolorbox}
\tcbuselibrary{skins,breakable}
\usepackage{hyperref}
\usepackage{mathtools}
\DeclarePairedDelimiter\ceil{\lceil}{\rceil}
\DeclarePairedDelimiter\floor{\lfloor}{\rfloor}

\usepackage{dsfont}

\hypersetup{
    colorlinks=true,
    linkcolor=blue,
    filecolor=magenta,      
    urlcolor=cyan,
    bookmarks=true,
    pdfauthor=Thaqib,
    bookmarksopen=true
}

\newtcbtheorem{defn}{Definition}{
    width=\textwidth,
    colback=white!20,
    colframe=orange,
    colbacktitle=orange,
    fonttitle=\bfseries,
    sharp corners,
    boxrule=1pt,
    breakable,
    enhanced,
    boxed title style={sharp corners},
    attach boxed title to top left
}{def}



\newtcbtheorem{axm}{Axiom}{
    width=\textwidth,
    colback=white!20,
    colframe=black,
    colbacktitle=black,
    fonttitle=\bfseries,
    sharp corners,
    boxrule=1pt,
    breakable,
    enhanced jigsaw,
    boxed title style={sharp corners},
    attach boxed title to top left
}{axm}

\newtcbtheorem{prop}{Proposition}{
    width=\textwidth,
    colback=white!20,
    colframe=black,
    colbacktitle=black,
    fonttitle=\bfseries,
    sharp corners,
    boxrule=1pt,
    breakable,
    enhanced jigsaw,
    boxed title style={sharp corners},
    attach boxed title to top left
}{prop}

\newtcbtheorem{thm}{Theorem}{
    width=\textwidth,
    colback=deepblue!2,
    colframe=deepblue,
    colbacktitle=deepblue,
    fonttitle=\bfseries,
    sharp corners,
    boxrule=1pt,
    breakable,
    enhanced,
    boxed title style={sharp corners},
    attach boxed title to top left
}{thm}

\newtcbtheorem{lemm}{Lemma}{
    width=\textwidth,
    colback=deepred!0,
    colframe=deepred,
    colbacktitle=deepred,
    fonttitle=\bfseries,
    sharp corners,
    boxrule=1pt,
    breakable,
    enhanced,
    boxed title style={sharp corners},
    attach boxed title to top left
}{lemm}



\newtcbtheorem{coll}{Corollary}{
    width=\textwidth,
    colback=white!20,
    colframe=dkgreen,
    colbacktitle=dkgreen,
    fonttitle=\bfseries,
    sharp corners,
    boxrule=1pt,
    breakable,
    enhanced,
    boxed title style={sharp corners},
    attach boxed title to top left
}{thm}




\usepackage{setspace}
\setstretch{1.7}
\usepackage{graphicx}
\usepackage[left=2cm,right=2cm,top=2cm,bottom=2cm]{geometry}

\usepackage{listings}
\usepackage{color}
\definecolor{dkgreen}{rgb}{0,0.3,0}
\definecolor{gray}{rgb}{0.5,0.5,0.5}
\definecolor{mauve}{rgb}{0.58,0,0.82}

\definecolor{deepblue}{rgb}{0,0,0.5}
\definecolor{deepred}{rgb}{0.6,0,0}
\definecolor{deepgreen}{rgb}{0,0.5,0}
\lstset{frame=tb,
  language=python,
  aboveskip=2mm,
  belowskip=2mm,
  showstringspaces=false,
  columns=flexible,
  basicstyle={\linespread{0.9}\small	tfamily},
  numbers=none,
  numberstyle=	iny\color{gray},
  keywordstyle=\color{blue},
  commentstyle=\color{dkgreen},
  stringstyle=\color{deepred},
  breaklines=true,
  breakatwhitespace=true,
  tabsize=4
}

\theoremstyle{definition}
\newtheorem{definition}{Definition}[section]
\newtheorem{example}{Example}

\newtheorem{theorem}{Theorem}[section]
\newtheorem{corollary}{Corollary}[theorem]
\newtheorem{lemma}[theorem]{Lemma}


\newcommand{\ang}[1]{\langle #1 \rangle}

\newcommand{\OR}{\vee}

\newcommand{\AND}{\wedge}

\author{Thaqib Mo.}
\title{ Elementary Number Theory }
\begin{document}
\maketitle
\newpage
\section{Integral Domains}
\begin{defn}{Integral Domains}{}
Let $R$ be a commutative ring, then $a\in R$ is called the \emph{zero divisor}, if there is some $b\in R$ with $b\neq 0$ for which $ab=0$.  
\\
An \emph{Integral Domain} is a commutative ring $R$, with $R\neq \{0\}$ such that $0$ is the only \emph{zero divisor}. If we have $ab=0$ then either $a=0$ or  $b=0$. 
\end{defn}

We can define Integral Domains in another equivalent way using the "cancellation law". 
\begin{thm}{}{}
A commutative ring $R\neq \{0\}$ is an integral domain if and only if for all $a,b,c \in R$ if $a\neq 0$ and 
\[ab=ac\]
Then
\[b=c\] 
\end{thm} 
\begin{proof}
Suppose $R$ is an integral domain, and we have $ab=ac$ and $a\neq 0$ then $ab-ac = 0$ and then $a(b-c) = 0$. Since $R$ is an integral domain we must have $b-c = 0$ that implies $b=c$. 
\\
Now suppose $R$ is a ring where the commutative property holds. Assume we have $ab=0$ If $a=0$ we are done, suppose $a\neq 0$ then 
\[ab=a\cdot 0 \leadsto b=0\]   
\end{proof}

\begin{example}
The ring $\mathbb{Z}$ is an integral domain.  
\end{example}

\begin{example}
The commutative rings $\mathbb{Q}, \mathbb{R}$ are an integral domains.  
\end{example}

The rings $\mathbb{Q}$ and $\mathbb{R}$ are more than rings. They are also \emph{fields}. 
\begin{defn}{Fields}{}
A ring $F$ is called a \emph{field} if it is commutative, and if every non zero element in $F$ has a multiplicative inverse. That means if $a\in F$ with $a\neq 0$ then we have $b\in F$ such that 
\[ab=1\]  
\end{defn}
For the fields $\mathbb{Q,R}$ if we have $r\in \mathbb{Q}$ then we also have $\frac{1}{r}\in \mathbb{Q}$ and $r\cdot \frac{1}{r} = 1$. The same applies for the field $\mathbb{R}$. The ring $\mathbb{Z}$ is not a field since not every element has a multiplicative inverse. 

\newpage
\begin{thm}{}{}
Every subring of a field is an integral domain. In particular, every field is an integral
domain.
\end{thm}
\begin{proof}
Let $F$ be a field and $R$ be a sub ring. Since $\times$ in $R$ and $\times$ in $F$ is the same, $(R,\times)$ is commutative and $R$ is a commutative ring. Now suppose we have $a,b\in R$ such that $ab=0$. If $a=0$ we are done. Assume $a\neq 0$ since this equation also holds in $F$ then there is some $a^{-1}\in F$ such that $aa^{-1}=1$ then we get 
\begin{align*}
&ab=0\\
&aba^{-1}=0a^{-1}\\
&b=0
\end{align*}
\end{proof}

\begin{example}
If $n\geq 2$ is composite then $\mathbb{Z}/n\mathbb{Z}$ is not an integral domain. Since there is a factorization of $n=ab$ then $[a],[b]$ are both non zero elements with $[a][b]=[ab]=[n]=[0]$
\end{example}


\begin{example}
We define the ring of \emph{Gaussian integers} denoted by $\mathbb{Z}[i]=\{a+bi : a,b\in \mathbb{Z}\}$ where addition is given by 
\[a+bi+c+di = (a+b)+(c+d)i\]
and multiplication is given by 
\[(a+bi)(c+di) = (ac-bd)+(ad+bc)i\]
This is a subring is $\mathbb{C}$ the complex numbers. 
\end{example}

\subsection{Basic Properties of Integral Domains}

\begin{thm}{}{}
If $R$ is an integral domain then $\mathrm{Char} R = 0$ or $\mathrm{Char} R$ is prime. 
\end{thm}
\begin{proof}
Suppose $R$ is an integral domain and $\mathrm{Char}R\neq 0$ and $\mathrm{Char}R$ is not prime. Then if we have $\mathrm{Char} R=1$, then $R$ is the zero ring since $1=0$, which is not possible due to the definition of integral domain. Now suppose $\mathrm{Char}R=n$ where $n>1$ is not prime. Then we have $n=ab$ then $a\cdot 1_R$ and $b\cdot 1_R$ are non zero elements but $(a\cdot 1)(b\cdot 1) = ab\cdot 1 =0$ that contradicts the definition of an integral domain. 
\end{proof}
Note that this again shows that $\mathbb{Z}/n\mathbb{Z}$ is not an integral domain

\newpage
\begin{thm}{}{}
Every finite integral domain is a field.
\end{thm}

\begin{proof}
Let $R$ be an integral domain, and suppose $|R|=n$. Let $a\in R$ with $a\neq 0$ consider the multiplication map $\phi_a(r) = ar$ then $\phi$ is injective since if we have $\phi(r)=\phi(s)$ then $ra=sa$ since $R$ is an integral domain we can use the cancellation property to get $r=s$.
\\
So we have an injective function $\phi:R\rightarrow R$. Since $R$ is finite then this implies $\phi$ is surjective. Given that $\phi$ is injective we have $|\phi(R)|=n$. Since $\phi$ is surjective there must be some $b\in R$ such that $\phi(b) = 1$ which means $ab=ba = 1$ thus $a$ has an multiplicative inverse in $R$. By definition $R$ is a field.  
\end{proof}


\subsection{Divisibility and Associates}
\begin{defn}{Divisibility arbitrary integral domain}{}
Let $R$ be an integral domain, and let $a,b\in R$ we say $a$ divides $b$ and denote it by $a\mid b$ if there is some $c\in R$ such that $b=ac$
\end{defn}
For example, consider the Gaussian integers $\mathbb{Z}[i]$ and we say that $2+i$ divides $5$ since $5=(2+i)(2-i)$. If $F$ is a field and $a\in F$ with $a\neq 0$ then $a\mid b$ for any $b\in F$ since $b=a(a^{-1}b)$. 

\begin{prop}{}{}\label{prop1}
Let $R$ be an integral domain
\begin{itemize}
\item[(1)] For all $a\in R$, we have $a\mid a$. 
\item[(2)] If $a,b,c\in R$ such that $a\mid b$ and $b\mid c$ then $a\mid c$. 
\item[(3)] If $a,b,c\in R$ such that $a\mid b$ and $a\mid c$ then $a\mid (bx+cy)$ for all $x,y\in R$. 
\end{itemize}
\end{prop}
\begin{proof}
\begin{itemize}
\item[(1)] Since $a=1\cdot a$ that means $a\mid a$. 


\item[(2)] Given $a\mid b$ we know that $b=ak$ for some $k\in R$ and we also have $c=b\ell$ for some $\ell \in R$. Then 
\[c=b\ell = (ak)\ell = a(k\ell)\]
That means we have $a\mid c$. 

\item[(3)] We know that $b=ak$ and $c=a\ell$ for any $x,y\in R$ we have 

\[bx+cy = akx+a\ell y = a(kx+\ell y)\]

Since $kx+\ell y\in R$ we have $a\mid bx+cy$
\end{itemize}
\end{proof}
\newpage

\paragraph{Note} since the relation is reflexive and transitive we can define an equivalence relation $a\sim b$ if $a\mid b$ and $b\mid a$. if we have $a\sim b$ then we say $a,b$ are \emph{associate} in $R$. 
\\
We can also make another order relation on the set of equivalence classes under $\sim$ by  $[a]_\sim \mid [b]_\sim$ if $a\mid b$. This is well defined and the choice of representative does not matter.  

\begin{defn}{}{}
Let $R$ be a ring and then $r\in R$ is called \emph{unit} of $R$ if $r$ has multiplicative inverse in $R$. The set of all unit of $R$ is denoted by $R^\ast$.  
\end{defn}
This is same as the the group of units of the monoid $(R,\times)$. 

\begin{thm}{}{}
Let $R$ be an integral domain. Given $a,b\in R$ we have $a\sim b$ if and only if $a=ub$ for some $u\in R^\ast$
\end{thm}\label{thm5}

\begin{proof}
($\Rightarrow$) First assume, $a\sim b$ then we have $a\mid b$ and $b\mid a$ so we have $b=ak$ and $a=b\ell$ this leads to 

\[b=ak=(b\ell)k=b(\ell k)\]
If we have $b=0$ then $a=b\ell = 0\ell = 0$ so we have $a=1\cdot b$ and we know that $1\in R^\ast$. Now consider the case $b\neq 0$, then $b\cdot 1 = b(\ell k)$ applying the cancellation property we get $1 = \ell k$ this means we have $\ell \in R^\ast$ so $a=\ell b$ where $\ell \in R$. 
\\
($\Leftarrow$) Suppose $a=ub$ this implies $b\mid a$ for some $u\in R^*$. Multiplying both sides by $u^{-1}$ gives $u^{-1}a=u^{-1}ub\leadsto b=u^{-1}a$ so we have $a\mid b$.

\end{proof}
We can apply this to the ring $\mathbb{Z}$ and we get $a\sim b$ if and only if $a=b$ or $a=-b$. 

\newpage
\section{Division with Remainder and Greatest Common Divisor}
\subsection{Division with Remainder in $\mathbf{\mathds{Z}}$}

\begin{thm}{Quotient and Remainder in $\mathds{Z}$}{}
Let $a,b \in \mathbb{Z}$, with $b>0$. Then there exists \emph{unique} integers $q,r$ with $0\leq r < b$ such that 
\[a=\left(b\times\underbrace{q}_{\text{quotient}}\right)+\overbrace{r}^{\text{remainder}}\]
\end{thm}
\begin{proof}
There are 2 cases. First let $a\geq 0$ and consider the set 
\[S=\{n\in \mathbb{N} : n = a-bq \text{ for some $q\in \mathbb{Z}$}\}\]

$S$ is non empty since $a=a-b(0)$ so we have $a\in S$. So by the Well ordering principle $S$ has a least element. Let $r$ be the least element of $S$. Then $r=a-bq \leadsto a=bq+r$. We need to check if $0\leq r<b$, we have $r\geq 0$ since $r$ is a natural number. Now assume $r\geq b$ then $r-b\geq 0$ that means $r-b = a-bq-b = a-b(q+1)$ that means we have $r-b\in S$ which is a contradiction since $r$ was the least element. Thus we have $0\leq r < b$. 
\\

Otherwise , if $a<0$ then $-a>0$ and the first part gives $q_0,r_0 \in \mathbb{Z}$ such that $-a=bq_0 + r_0$. Now if we have $r_0 = 0$ then $-a=bq_0 \leadsto a=b(-q_0)$. Otherwise if we have $r\neq 0$ then we can write

\begin{align*}
& a= b(q_0)-r_0 = b(q_0)-b+b-r_0 \\
&= b(q_0 - 1) + b-r_0
\end{align*}
We have $q= q_0 -q$ and $r=b-r_0$ both in $\mathbb{Z}$ and $0<b-r_0 < b$. we have proven the existence of $r,q$ for all $a\in \mathbb{Z}$. 
\\


To prove \emph{uniqueness} consider $q',r'$ such that $a=bq'+r'$ we have 

\[r+bq = r'+bq'\]
This means we have $r-r' = b(q-q')$ if we have $q=q'$ then $r=r'$ and we are done. Otherwise , if $q\neq q'$ taking the absolute values of both sides
\[|r-r'| = |b||q-q'|\geq b\]

But $r,r'$ are both positive and strictly less than $b$, so $|r-r'|\geq b$ is a contradiction. So we must have $q'= q \rightarrow r=r'$. 
\end{proof}
\newpage
\subsection{Division with Remainder in $\mathds{Z}[i]$} 

\begin{defn}{Norm}{}
For $a+bi \in \mathbb{Z}[i]$ we define \emph{norm} of $a+bi$ written as $N(a+bi)$ to be $a^2 + b^2 \in \mathbb{N}$
\end{defn}
\begin{example}
Suppose we want to divide $2+i$ by $1+i$ with remainder then we must have 
\[2+i = (1+i)\gamma + \delta\]
With $0\geq N(\delta) <N(1+i)$. First we know that 
\[\frac{2+i}{1+i} = \frac{3}{2}-\frac{1}{2}i\]
Now we have 4 choices to round this up to the nearest integer. We can so $\frac{3}{2}\rightarrow 2$ or $1$ and for $\frac{-1}{2}$ we can do 0 or -1. Lets assume we take $\gamma = 2+0i$ then the remainder is 
\[(2+i)-(1+i)2 = -i\]
This leads to $(2+i) = (1+i)2 + (-i)$ this remainder is valid since $N(-i) = 1<N(1+i)$. We need to show that this works in general. 
\end{example}
\begin{thm}{Division with remainder in Gaussian integers}{}
Let $\alpha, \beta \in \mathbb{Z}[i]$ then there exists $\gamma, \delta \in \mathbb{Z}[i]$ such that 
\[\alpha = \beta \gamma + \delta\]
\end{thm}
\begin{proof}
Let $\alpha = a+bi$ and $\beta = c+di$ performing division in $\mathbb{C}$ we get
\[\frac{a+bi}{c+di} = \frac{(ac+bd) + (bc-ad)i}{c^2 + d^2} = r+si\]

Now we choose $m,n\in \mathbb{Z}$ such that $|m-r|\leq \frac{1}{2}$ and $|n-s|\leq \frac{1}{2}$. Then let $\gamma = m+ni\in \mathbb{Z}[i]$ and set $\delta = \alpha - \beta \gamma$, so we have $\alpha = \beta \gamma + \delta$ but we need to show $0\leq N(\delta)<N(\beta)$. We have $0\leq N(\delta)$ holds by definition now consider
\[\delta = \alpha - \beta \gamma = \beta (r+si) - \beta (m+ni) = \beta((r-m)+(s-n)i)\]

 Taking the complex absolute value and squaring $N(\delta) = |\delta|^2 = |\beta|^2|(r-m)+(s-n)i|^2 = N(\beta)((r-m)^2+(s-n)^2)$ 
Since we fixed $|r-m|\leq \frac{1}{2}$ and $|s-n|\leq \frac{1}{2}$ so we get $N(\delta) \leq N(\beta)/2<N(\beta)$

\end{proof}
\newpage
In division with remainder in $\mathbb{Z}[i]$ we loose the uniqueness property. Since we could have choose from 4 possible values of $\gamma$ which leads to valid values for $\delta$. 

\begin{defn}{}{}
Let $R$ be an integral domain, $R$ has a \emph{division algorithm} if there exists a function 
\[d:R\setminus \{0\}\rightarrow \mathbb{N}\] 
called the \emph{divisor function} such that for $a,b \in R$ with $b\neq 0$ we have $q,r\in R$ such that 
\[a=bq+r\]
Then either $d(r)<d(b)$ or else $r = 0$
\end{defn}
We have proven that both $\mathbb{Z}[i], \mathbb{Z}$ have division algorithm with divisor functions $d(\alpha) = N(\alpha)$ and $d(\alpha) = |\alpha|$ for integers. 

\subsection{Greatest Common Divisor}
Given integers $a,b$ we need to find $d$ dividing both $a,b$ and we need to choose the largest such integer with this property. This can be generalized to any integral domain. 

\begin{defn}{Greatest Common Divisor}{}
Let $R$ be an integral domain let $a,b\in R$ with $a,b\neq 0$ and element $d\in R$ is called the greatest common divisor ($\gcd (a,b)$) if: 
\begin{itemize}
\item[\#1] $d\mid a$ and $d\mid b$
\item[\#2] If $f\in R$ another common divisor of $a,b$ such that $f\mid a$ and $f\mid b$ then $f\mid d$
\end{itemize}
\end{defn}
The $\gcd$ of 2 elements may not be unique, rather it picks out a unique equivalence class with respect to the associate relation $\sim$. We can see that in the next theorem. 

\begin{thm}{}{}
Let $R$ be an integral domain. Let $a,b \in R$ with $a,b\neq 0$. If $d_1$ and $d_2$ are both greatest common divisor of $a,b$ then $d_1\sim d_2$. Conversely if $d_1$ is a greatest common divisor  of $a,b$ then $d_1\sim d_2$ where $d_2$ is another greatest common divisor.  
\end{thm}
\begin{proof}
Suppose $d_1, d_2$ are both greatest common divisor of $a,b$. Since $d_1$ is a common divisor of $a,b$ and $d_2$ is the $\gcd$ we must have $d_1\mid d_2$. By symmetry we also have $d_2\mid d_1$ by definition of the relation we have $d_1\sim d_2$
\\
\newpage
Now assume, $d_1$ is a $\gcd$ of $a,b$ and that $d_2 \mid d_1$. Then $d_1\mid d_2$ and $d_2\mid d_1$. The transitive property gives $d_2\mid a$ and $d_2\mid b$. So $d_2$ is a common divisor of $a,b$. Now assume $e$ another common divisor of $a,b$ we must have $e\mid d_1$ again by transitivity we get $e\mid d_2$. Thus $d_2$ is a $\gcd$ of $a,b$ by definition. 
\end{proof}
So $\gcd$ is not unique in integral domain. It picks out a unique equivalence class $R/\sim$. Then by theorem \hyperref[thm5]{Theorem 5}, if $d$ is a greatest common divisor it takes the form $ud$ for some chosen $u\in R^*$. So we can use the notation $\gcd(a,b)$ do denote the equivalence class of gcds.  
\\
For example in the ring $\mathbb{Z}$ if $d$ is one gcd of $a,b$ then do is $-d$ and it is the only other $gdc$. 

\section{The Euclidean Algorithm}
\begin{lemm}{}{}
Let $R$ be an integral domain. Suppose $a,b,q,r\in R$ such that 
\[a=bq+r\] 
Then some $d\in R$ is $\gcd$ of $a,b$ if and only if it is the $\gcd$ of $b,r$. That is 
\[\gcd(a,b)\sim \gcd(b,r)\]
\end{lemm}
\begin{proof}
Suppose $\gcd(a,b) = d$ then we have $d\mid a$ and $d\mid b$ it follows from \hyperref[prop1]{Proposition 1} that 
\[d\mid a+b(-q)\] 
This is $d\mid r$ so $d$ is a common divisor of both $b,r$. Now suppose $e$ is a common divisor of $b,r$ then $e\mid b$ and $e\mid b$ again we have 
\[e\mid b\cdot q + r\]
So we have $e\mid a$ since $e$ is a common divisor of both $a,b$ then we have $e\mid d$. The same logic follows for the $(\Leftarrow)$ case. 
\end{proof}
Now if we have a Integral Domain $R$ with divisor function $D$ we can use the above lemma to compute the $\gcd$ of $a,b\neq 0$. First we have 
\[a=bq_0 + r_1\]
When $q_0,r_1\in R$ by definition we have $r_1= 0$ or $D(r_1) < D(b)$. If we have $r_1 = 0$ then we get $b\mid a$, then let $e$ be another common divisor of $a,b$ we have $e\mid a$ and $e\mid b$ by definition so $b=\gcd(a,b)$. Otherwise we use the lemma to get $\gcd(a,b)\sim \gcd(b,r_1)$ so the task is down to finding the $\gcd$ of $b,r_1$. \newpage
We can repeat the procedure to get
\[b=r_1q_1 + r_2\] 
Again either $r_2 = 0$ then $r_1$ is the $\gcd$ or $D(r_2)<D(r_2)$ we again have $\gcd(b,r_1)\sim \gcd(r_1,r_2)$ we continue this process until we get a $0$ remainder. So to outline this process we have 
\begin{align*}
a&=q_0b+r_1 \\
b &= r_1q_1 + r_2 \\
r_1 &= r_2q_2+r_3 \\
&\vdots \\
r_{n-2} &= r_{n-1}q_{n-1} + r_{n} \\
r_{n-1} &= r_{n}q_{n} + 0
\end{align*}
Since the sequence $D(r_1)>D(r_2)>D(r_3)>\cdots>D(r_{n-1})$ is a strictly decreasing sequence of natural numbers, it is bounded below by 0 and it must reach a zero remainder at some point. Using the lemma we get $\gcd(a,b) = r_{n}$, this leads to the following theorem. 

\begin{thm}{}{}\label{thm9}
Let $R$ be an integral domain with a division algorithm. Given any two non zero elements $a,b\in R$ $\gcd(a,b)$ exists.  
\end{thm}

\subsection{Extended Euclidean Algorithm}
We can utilize the euclidean algorithm to compute the solutions to linear equations. Suppose we have an integral domain $R$ with division algorithm and non-zero $a,b\in R$. Suppose we have $\gcd(a,b) = d$ we can find $x,y\in R$ such that 
\[ax+by=d\]
After running the euclidean algorithm we get  
\begin{align*}
a&=q_0b+r_1 \\
b &= r_1q_1 + r_2 \\
r_1 &= r_2q_2+r_3 \\
&\vdots \\
r_{n-2} &= r_{n-1}q_{n-1} + r_{n} 
\end{align*}
With the last step with zero remainder left out. 
\newpage
We can reverse the order of these equations and isolate the remainder in each one to get
\begin{align*}
r_n &= r_{n-2}-r_{n-1}q_{n-1} \\
r_{n-1} &= r_{n-3}-r_{n-2}q_{n-2} \\
&\vdots\\
r_2 &= b-r_1q_1 \\
r_1 &= a-q_0b
\end{align*}
We focus on the first 2 equations 
\begin{align*}
r_n &= r_{n-2}-r_{n-1}q_{n-1} \\
r_{n-1} &= r_{n-3}-r_{n-2}q_{n-2} 
\end{align*}
We can substitute $r_{n-1}$ from the second equation in the first equation to get an equation in the form $r_{n-1} = r_{n-2}x+r_{n-3}y$ then we can use RHS from the third equation for $r_{n-2}$ to get $r_{n-3}x+r_{n-4}y$ then we keep on repeating this until the final equation. This leads to the following theorem 
\begin{thm}{}{}\label{thm10}
Let $R$ be in integral domain with a division algorithm and let $a,b \in R$ with $a,b\neq 0$. If $d=\gcd(a,b)$ then there exists $x,y\in R$ such that 
\[ax+by = d\]  
\end{thm}
\newpage
\section{Linear Diophantine Equations and Linear Congruences}
\subsection{Linear Diophantine Equations in Two Variables}
Suppose $R$ is an integral domain with a division algorithm given $a,b,c\in R$ a  Linear Diophantine Equations in Two Variables is an equation of the form 
\[ax+by = c\]
With $x,y \in R$, there are two main questions
\begin{itemize}
\item[(1)] Does a solution to the equation exist?
\item[(2)] If yes, can we find \textbf{all} the solutions?
\end{itemize} 
\begin{thm}{}{}
Let $a,b,c\in R$ where $R$ is an integral domain with both $a,b$ not being zero. If there is a solution to the equation 
\[ax+by = c\]
Then 
\[\gcd(a,b)\mid c\] 
\end{thm}
\begin{proof}
Let $d=\gcd(a,b)$, by definition $d\mid a$ and $d\mid b$ so we must have 
\[d\mid ax+by \leadsto d\mid c\]
\end{proof}

\begin{thm}{}{}
Let $a,b,c\in R$ where $R$ is an integral domain with both $a,b$ not being zero. Suppose $\gcd(a,b)\mid c$ then the equation 
\[ax+by = c\]
has a solution with $x,y\in R$
\end{thm}
\begin{proof}
Let $d=\gcd(a,b)$ this exists, by \hyperref[thm9]{Theorem 9}. Moreover by \hyperref[thm10]{Theorem 10} there exists $x_0,y_0 \in R$ such that
\[ax_0 + by_0 = d\]
Since $d\mid c$ we have $c=kd$ we get 
\[c=kd = k(ax_0 + by_0) = a(kx_0)+b(ky_0)\]
\end{proof}
\subsection{Divisibility Results}
To solve linear Diophantine equations, it is necessary to establish a few results on divisibility.

\begin{thm}{}{}
Suppose $R$ is an integral domain with a division algorithm suppose we are given $a,b,c\in R$. If $a\mid bc$ and $1\sim \gcd(a,b)$ then $a\mid c$
\end{thm}
\begin{proof}
Given $a\mid bc$ we know that $bc=ak$, then since $\gcd(a,b) = 1$ we have $x,y$ such that 
\[ax+by = 1\]
Then we have 
\[acx+bcy=c\]
Now since $bc = ak$ we get
\begin{align*}
acx+aky &= c\\
a(cx+ky) & = c
\end{align*}
Which means $a\mid c$
\end{proof}
\begin{lemm}{}{}\label{lemm2}
Let $R$ be an integral domain with division algorithm and suppose we have $a,b\in R$ not both zero. Suppose $d=\gcd(a,b)$ then we have 
\[a=a_0d \;,\; b=b_0d\]
Then $\gcd(a_0,b_0) \sim 1$  
\end{lemm}
\begin{proof}
The equation 
\[ax+by = d\]
always has a solution so we have
\[a_0dx+b_0dy = d\]
This leads to 
\[a_0x+b_0y = 1\]
This means $\gcd(a_0,b_0)\mid 1$ and $1\mid \gcd(a_0,b_0)$ holds. So we have $\gcd(a_0,b_0) \sim 1$ .  
\end{proof}
\newpage
\subsection{The General Solution of a Linear Diophantine Equation}
\begin{thm}{}{}
Let $R$ be an integral domain with a division algorithm. Let $a,b,c\in R$ such that both $a,b$ are not zero and let $d=\gcd(a,b)$. Assume that $d\mid c$. Also we write $a=a_0d$ and $b=b_0d$ then the equation 
\[ax+by=c\]
complete set of solutions are
\[(x,y) = (x_0+kb_0,y_0-ka_0)\]
Where $k\in R$ is arbitrary and $(x_0,y_0)$ is a particular solution.    
\end{thm}
\begin{proof}
Let $(x_0,y_0)$ be a particular solution to $ax+by=c$ which exists since $\gcd(a,b)\mid c$. Suppose we have another solution $(x_1,y_1)$ then we know that 
\begin{align*}
ax_0+by_0 = c \\
ax_1+by_1 = c
\end{align*}
Then we get 
\[a(x_1-x_0)+b(y_1-y_0) = 0\]
We have $a=a_0d$ and $b=b_0d$ 
\[a_0d(x_1-x_0)+b_0d(y_1-y_0) = 0\]
Applying the cancellation law in the integral domain leads to 
\[a_0(x_1-x_0)=-b_0(y_1-y_0) \]
Then we have $b_0\mid a_0(x_1-x_0)$, but we have $\gcd(a_0,b_0)\sim 1$ by \hyperref[lemm2]{Lemma 2} so we get $b_0\mid (x_1-x_0)$ this means we have 
$x_1-x_0 = kb_0$ this means we have $x_1 = x_0 + kb_0$. Using this substitution we have 
\[a_0kb_0 = -b_0(y_1-y_0)\]
This means we have $ka_0 = y_0 - y_1$  this gives $y_1 = y_0-ka_0$. So we have if $(x_0,y_0)$ is a solution then so is $(x_0+kb_0,y_0-ka_0)$. 
\\
Conversely we can also check that every ordered pair $(x_1,y_1)=(x_0+kb_0,y_0-ka_0)$ is a solution then we have 
\[ax_1+by_1=a(x_0+kb_0)+b(y_0-ka_0) = ax_0+by_0+k(ab_0-ba_0) = c+k(da_0b_0 -db_0a_0) = c\] 
\end{proof}
\subsection{Multiplicative Inverses in $\mathds{Z}\mathbf{/n}\mathds{Z}$}
Using Diophantine equation we can construct a procedure for calculating multiplicative inverse of an element when it exists in $\mathbb{Z}/n\mathbb{Z}$. Suppose we have $[a]\in \mathbb{Z}/n\mathbb{Z}$ Then some $[x]$ is the inverse if and only if $[a][x] = [1]$ this means we have 
\[ax\equiv 1\bmod n\]
This is equivalent to $n\mid 1-ax$ this means we have $1-ax = ny$ so we have 
\[ax+ny=1\]
Thus finding multiplicative inverse we can find the inverse. Moreover it exists if and only if $\gcd(a,n) = 1$. In the case where $n=p$ is prime then $[a]\in \mathbb{Z}/p\mathbb{Z}$ such that $[a]\neq [0]$, then $\gcd(a,p) = 1$, because we have $a\nmid p$  so there are no common divisors other than 1. This proves that every non-zero element in $\mathbb{Z}/p\mathbb{Z}$ has an inverse therefore it is a field. 
\begin{example}
Suppose we want to find inverse of $[5]\in \mathbb{Z}/13\mathbb{Z}$ this means we have to solve 
\[5x+13y=1\]
So we run the euclidean algorithm to get
\begin{align*}
13 &= 5(2) + 3 \\
5 &= 3+2 \\
3 &= 2+1 \\
2 &= 1\cdot 2
\end{align*} 
So we have $\gcd(5,13) = 1$ and $[5]^{-1}$ exists. Now we can use the back substitution to get
\begin{align*}
&1 = 3-2 \\
&2 = 5-3 \\
&3 = 13-5(2)\\
\end{align*} 
Making the substitution we get
\begin{align*}
1&=3-2\cdot 1\\
&= 3-(5-3)\cdot 1\cdot 1 \\
&=3\cdot 2-5 \\
&= (13-5(2))\cdot 2-5\cdot 1\\
&= 13\cdot 2 - 5\cdot 5
\end{align*} 
So the solution is $x=-5$ so we have $[5]^{-1}=[-5] = [8]$
\end{example}




























 






















\end{document}
