\documentclass[16pt,a4paper]{article}
\usepackage[utf8]{inputenc}
\usepackage{amsmath}
\usepackage{amsfonts}
\usepackage{amssymb}
\usepackage{amsthm}
\usepackage{physics}


\usepackage[many]{tcolorbox}
\tcbuselibrary{skins,breakable}
\usepackage{hyperref}
\usepackage{mathtools}
\DeclarePairedDelimiter\ceil{\lceil}{\rceil}
\DeclarePairedDelimiter\floor{\lfloor}{\rfloor}

\usepackage{dsfont}

\hypersetup{
    colorlinks=true,
    linkcolor=blue,
    filecolor=magenta,      
    urlcolor=cyan,
    bookmarks=true,
    pdfauthor=Thaqib,
    bookmarksopen=true
}

\newtcbtheorem{defn}{Definition}{
    width=\textwidth,
    colback=white!20,
    colframe=orange,
    colbacktitle=orange,
    fonttitle=\bfseries,
    sharp corners,
    boxrule=1pt,
    breakable,
    enhanced,
    boxed title style={sharp corners},
    attach boxed title to top left
}{def}


\newtcbtheorem{algo}{Algorithm}{
    width=\textwidth,
    colback=white!20,
    colframe=mauve,
    colbacktitle=mauve,
    fonttitle=\bfseries,
    sharp corners,
    boxrule=1pt,
    breakable,
    enhanced jigsaw,
    boxed title style={sharp corners},
    attach boxed title to top left
}{algo}


\newtcbtheorem{axm}{Axiom}{
    width=\textwidth,
    colback=white!20,
    colframe=black,
    colbacktitle=black,
    fonttitle=\bfseries,
    sharp corners,
    boxrule=1pt,
    breakable,
    enhanced jigsaw,
    boxed title style={sharp corners},
    attach boxed title to top left
}{axm}

\newtcbtheorem{prop}{Proposition}{
    width=\textwidth,
    colback=white!20,
    colframe=black,
    colbacktitle=black,
    fonttitle=\bfseries,
    sharp corners,
    boxrule=1pt,
    breakable,
    enhanced jigsaw,
    boxed title style={sharp corners},
    attach boxed title to top left
}{prop}

\newtcbtheorem{thm}{Theorem}{
    width=\textwidth,
    colback=deepblue!2,
    colframe=deepblue,
    colbacktitle=deepblue,
    fonttitle=\bfseries,
    sharp corners,
    boxrule=1pt,
    breakable,
    enhanced,
    boxed title style={sharp corners},
    attach boxed title to top left
}{thm}

\newtcbtheorem{lemm}{Lemma}{
    width=\textwidth,
    colback=deepred!0,
    colframe=deepred,
    colbacktitle=deepred,
    fonttitle=\bfseries,
    sharp corners,
    boxrule=1pt,
    breakable,
    enhanced,
    boxed title style={sharp corners},
    attach boxed title to top left
}{lemm}



\newtcbtheorem{coll}{Corollary}{
    width=\textwidth,
    colback=white!20,
    colframe=dkgreen,
    colbacktitle=dkgreen,
    fonttitle=\bfseries,
    sharp corners,
    boxrule=1pt,
    breakable,
    enhanced,
    boxed title style={sharp corners},
    attach boxed title to top left
}{thm}




\usepackage{setspace}
\setstretch{1.7}
\usepackage{graphicx}
\usepackage[left=2cm,right=2cm,top=2cm,bottom=2cm]{geometry}

\usepackage{listings}
\usepackage{color}
\definecolor{dkgreen}{rgb}{0,0.3,0}
\definecolor{gray}{rgb}{0.5,0.5,0.5}
\definecolor{mauve}{HTML}{6a1b9a}

\definecolor{deepblue}{rgb}{0,0,0.5}
\definecolor{deepred}{rgb}{0.6,0,0}
\definecolor{deepgreen}{rgb}{0,0.5,0}
\lstset{frame=tb,
  language=python,
  aboveskip=2mm,
  belowskip=2mm,
  showstringspaces=false,
  columns=flexible,
  basicstyle={\linespread{0.9}\small	tfamily},
  numbers=none,
  numberstyle=	iny\color{gray},
  keywordstyle=\color{blue},
  commentstyle=\color{dkgreen},
  stringstyle=\color{deepred},
  breaklines=true,
  breakatwhitespace=true,
  tabsize=4
}

\theoremstyle{definition}
\newtheorem{definition}{Definition}[section]
\newtheorem{example}{Example}

\newtheorem{theorem}{Theorem}[section]
\newtheorem{corollary}{Corollary}[theorem]
\newtheorem{lemma}[theorem]{Lemma}


\newcommand{\ang}[1]{\langle #1 \rangle}

\newcommand{\OR}{\vee}
\newcommand{\Z}{\mathbb{Z}}
\newcommand{\C}{\mathbb{C}}
\newcommand{\R}{\mathbb{R}}
\newcommand{\Q}{\mathbb{Q}}

\newcommand{\fig}[2]{\begin{figure}[hbtp] 
 \centering
 \includegraphics[scale=#2]{figs/#1.png}
 \end{figure}
}



\newcommand{\AND}{\wedge}
\author{Thaqib Mo.}
\title{ Factoring Polynomials }
\begin{document}
\maketitle
\newpage
\section{Complex Numbers}
Using localization similar to the construction of $\mathbb{Q}$ we can construct elements of $\C$ in terms of ordered pairs $(a,b)\in \R\times \R$. The addition is defined component wise, 
\[(a,b) + (c,d) = (a+c,b+d)\]
and multiplication similar to the Gaussian integers is defined as
\[(a,b)\cdot (c,d) = (ac-bd, ad+bc)\]
The addition is the same as the addition in $\R$ so we can already conclude that $\C,+$ is an abelian group. $(1,0)$ is clearly the multiplicative identity and it is easy to check that multiplication is associative. 
\\
To check that every non-zero element in $\C$ has an multiplicative inverse: 
\begin{align*}
\frac{1}{a+bi} = \frac{a-bi}{a^2 + b^2} = \frac{a}{a^2+b^2} - \frac{b}{a^2+b^2}i
\end{align*}
So we have $\left( \frac{a}{a^2+b^2}, \frac{b}{a^2+b^2}\right)$ this ordered pair is the inverse of $(a,b)$. We have $a^2 + b^2 \neq 0$ when $(a,b)\neq (0,0)$. Now consider 
\begin{align*}
& (a,b)\cdot ((c,d)+(e,f)) = (a,b)\cdot (c+e,d+f) \\
& = \left(a(c+e) - b(d+f), a(d+f) + b(c+e)\right) \\
& = \left(ac- bd + ae -bf, ad+ bc +af +be\right) \\
& = (a,c)\cdot (e,f) + (a,b)\cdot (e,f)
\end{align*} 
So we can now say $\C$ is a field. 

\subsection{Complex Number Constructions and Properties}
\begin{defn}{}{}
Let $z\in\C$ and we write $z=a+bi$ for some $a,b \in \R$
\begin{itemize}
\item The form $a+bi$ is called the standard form of $z$ 
\item The number $a$ is the \emph{Real} part of $z$ denoted by $\Re(z)$ 
\item The number $b$ is the \emph{imaginary} part of $z$ denoted by $\Im(z)$
\item $\bar{z} = a-bi$ is called the \emph{complex conjugate}
\item $|z| = \sqrt{a^2 + b^2}$ is called the \emph{absolute value} of $z$
\end{itemize}
\end{defn}

\begin{prop}{}{}
$\phi:\C \rightarrow \C$ given by $\phi(z) = \bar{z}$ is a ring homomorphism. 
\end{prop}
\begin{proof}
Let $z=a+bi$ and $w=c+di$ \\
Consider $\phi(z+w) = \overline{z+w} = a+c -(b+d)i  = a-bi + c-di = \phi(z) + \phi(w)$. 
\\
$\phi(zw) = \bar{zw} = \overline{ac-bd + (ad+bc)i} = ac-bd -(ad+bc)i = ac-bd-adi-bci = (a-bi)(c-di) = \phi(z)\phi(w)$. For $\phi(1+0i) = 1-0i = 1$. So $\phi$ is a ring homomorphism. 
\end{proof}


\begin{prop}{}{}
For all $z\in \C$ we have $|zw| = |z||w|$ 
\end{prop}
\begin{proof}
Let $z=a+bi$ and $w=c+di$ we have $zw = ac-bd +(ad+bc)i$ so we have 
\begin{align*}
|zw| &= \sqrt{(ac-bd)^2 (ad+bc)^2} \\
& = \sqrt{a^2c^2 -2acbd + b^2d^2 + a^2d^2 +2adbc + b^2c^2}\\
&=\sqrt{a^2c^2  + b^2d^2 + a^2d^2 + b^2c^2}\\
&= \sqrt{a^2(c^2+d^2) + b^2(c^2+d^2)}\\
&= \sqrt{(a^2+ b^2)(c^2+d^2)}\\
&= \sqrt{a^2+b^2}\sqrt{c^2+d^2}\\
&= |z||w|
\end{align*}

\end{proof}

The \emph{triangle inequality} also holds in $\C$
\begin{thm}{Triangle inequality}{}
For all $z,w\in \C$ we have $|z+w|\leq |z| + |w|$
\end{thm}
\begin{proof}
\begin{align*}
|z+w|^2 &= (z+w)\overline{(z+w)}\\
&= (z+w)(\bar{z}+\bar{w}) \\
&= z\bar{z} + z\bar{w} + w\bar{z} + w\bar{w} \\
& = |z|^2 + |w|^2 + (z\bar{w} + \overline{z\bar{w}}) \\
& = |z|^2 + |w|^2 + 2\Re(z\bar{w})
\end{align*}
Note that $\Re(z\bar{w})\leq |z\bar{w}| = |z||\bar{w}| = |z||w|$. 
So we have \[|z+w|^2 = |z|^2 + |w|^2 + 2\Re(z\bar{w}) \leq |z|^2 + 2|z||w| + |w|^2 = (|z|+|w|)^2 \]
Taking the positive square roots gives the triangle in equality.   
\end{proof}
\subsection{Polar Form of a Complex Number}
\fig{fig1}{1}

So we write $a+bi = re^{i\theta}$ where $r$ is the magnitude and $\theta$ is the argument. There are be many values for the argument for the same complex number, in particular any $\theta + 2k\pi$ for $k\in \mathbb{Z}$ would work. 

\begin{thm}{}{}
For complex numbers $z_1 = r_1z^{\theta_1}$ and $z_2 = r_2z^{\theta_2}$ we have 
\[
z_1z_2 = r_1r_2e^{i(\theta_1 + \theta_2)}
\] 
 \end{thm} 
\begin{proof}
We have $z_1 = r_1e^{\theta_1} = r_1(\cos(\theta_1) + i\sin(\theta_1))$ and 
$z_2 = r_2(\cos(\theta_2) + i\sin(\theta_2))$ so we have 
\begin{align*}
z_1z_2 &= r_1r_2(\cos(\theta_1) + i\sin(\theta_1))(\cos(\theta_2) + i\sin(\theta_2)) \\
&= r_1r_2(\cos(\theta_1)\cos(\theta_2) -\sin(\theta_1)\sin(\theta_2) + i(\cos(\theta_1)\sin(\theta_1)+\sin(\theta_2)\cos(\theta_2))) \\
&= r_1r_2(\cos(\theta_1 + \theta_2) + i\sin(\theta_1\theta_2))\\
&= r_1r_2e^{i(\theta_1+\theta_2)}
\end{align*}
\end{proof}

\begin{coll}{(de Moivre's Theorem)}{}
Let $z$ be a complex number with $z\neq 0$. Then we have 
\[
z^n = r^ne^{n\theta}
\]
For $n\in \Z$
\end{coll}
\begin{proof}
The base case is trivial we have $z^0 = 1$ and $r^0e^{0\theta} = 1$. Now assume it holds true for some $n\in \mathbb{N}$. Now consider $n+1$
\begin{align*}
z^{n+1} &= zz^{n} \\
&= re^{\theta}r^{n}e^{n\theta} \\
&= rr^ne^{n\theta + \theta}\\
&=  r^{n+1}e^{(n+1)\theta}
\end{align*}
This completes the proof for $n\in \mathbb{N}$. Now consider $z^{-n}$ for $n\in \mathbb{N}$. By uniqueness of inverse since $\C$ is a field, we have $r^{-n}e^{-n\theta}z^n = 1$. 
\end{proof}
\section{The Fundamental Theorem of Algebra}

\subsection{Algebraically Closed Field}
\begin{thm}{Fundamental Theorem of Algebra}{}
Let $f\in \C[x]$ be a non-constant polynomial. Then $f$ has a root in $\C$. 
\end{thm}
This leads to the definition of algebraically closed fields

\begin{defn}{Algebraically Closed Field}{}
A field is algebraically closed if any non-constant polynomial $f\in F[x]$ has a root in $F$. 
\end{defn} 
 \newpage
Another equivalent way of formulating algebraically closed field is in the following theorem: 

\begin{thm}{}{}
A field is algebraically closed if and only if every non-constant polynomial $f\in F[x]$ can be factored as a product of linear polynomials. 
\[
f = c(x-a_1)(x-a_2) \cdots (x-a_n)
\]
Where $n=\deg f$ and $c,a_1,a_2,\ldots, a_n \in F$. 
\end{thm}

\begin{proof}
$(\Rightarrow)$ suppose that $F$ is algebraically closed. Consider the base case $n=1$, it is already in linear form so we are done. Now assume the results holds true for $\deg f = n$ and consider $\deg f = n+1$. 
\\
Since $F$ is algebraically closed $f$ has a root in $F$, and let $f(a_{n+1}) = 0$. Then by the factor theorem $(x-a_{n+1})$ divides $f$. So let 
\[
f = g(x-a_{n+1})
\] 
And since $\deg g = n$ by the inductive hypothesis it can be factored into linear factors and we have 
\[
f = c(x-a_{n+1})(x-a_1)(x-a_2) \cdots (x-a_n)
\] 
$(\Leftarrow)$ Suppose every non-constant polynomial in $F[x]$ can be factored into linear polynomials. Then 
\[
f = c(x-a_1)(x-a_2) \cdots (x-a_n)
\] 
Now $f$ always has a root in $f$ since $a_1, a_2, \ldots, a_n$ are all roots. 
\end{proof}
\newpage
\section{Irreducible Polynomials}

\begin{defn}{Irreducible and Reducible Polynomials}{}
Let $F$ be a field. Then $f$ is reducible if $f$ has a proper factorization that is $f=gh$ where $g,h\in F[x]$ and $\deg(g),\deg(h) \geq 1$. Otherwise $f$ is irreducible. 
\end{defn}
Using this definition every linear polynomial is automatically irreducible. For larger degrees we have:

\begin{prop}{}{}
  Let $F$ be a field and $f\in F[x]$ If $\deg f \geq 2$ be irreducible then $f$ has no roots. Conversely, if $\deg f = 2$ and $\deg f = 3$ has no roots then $f$ is irreducible. 
  \end{prop}  
\begin{proof}
Suppose we have $\deg f\geq 2$ is irreducible. Assume $f$ has a root in $F$. By the factor theorem we have $f = (x-c)h$ then we have $\deg	h = \deg f -1 \geq 1$ this means we have factored $f$ into 2 non-constant polynomials thus a contradiction. 
\\
Conversely suppose $\deg f = 2$ or $\deg 3 = f$ and that $f$ has no roots in $F$. Suppose $f$ is reducible in $F$ then we must have $f = gh$ and we must have $\deg f = \deg g + \deg h$ and this forces one of $\deg g, \deg h = 1$ either way $f$ has a linear factor and must have a root again a contradiction. Thus $f$ must be irreducible.  
\end{proof}
Note the converse does not generalize to higher degrees there can be polynomials $f$ with $\deg f = 4$ with no roots and still be irreducible. The above theorem also shows that $x^2 + 1 \in \R[x]$ is irreducible. 

\begin{coll}{}{}
If $F$ is algebraically closed and a non-constant polynomial $f\in F[x]$ is irreducible if and only if $\deg f = 1$
\end{coll}
\begin{proof}
We already know that a linear polynomial is irreducible. Conversely assume that $f\in F[x]$ is irreducible and $\deg f > 1$ then $f$ has no roots in $F$ by the above proposition, which is a contradiction to $F$ being algebraically closed. 
\end{proof}

\begin{example}
The polynomial $f(x) = x^3 +x+[1]\in \Z/2\Z$ is irreducible. A simple exhaustive proof can show this we have $f([0]) = [1]$ and $f([1]) = [1]$ so $f$ has no roots therefore $f$ is irreducible. 
\end{example}
\newpage
\subsection{Irreducible Polynomials in $\R[x]$}
\begin{lemm}{Conjugate Roots}{}
Suppose $f\in \R[x]$ and if $c\in \C$ is a root of $f$ then $\overline{c}$ is also a root. 
\end{lemm}
\begin{proof}
We know that $f(c) = 0$ so 
\begin{align*}
a_0 + a_1c + a_2c^2 + \ldots + a_nc^n = 0 
\end{align*}
Then we have 
\begin{align*}
0 &= \overline{0} \\
&= \overline{a_0 + a_1c + a_2c^2 + \ldots + a_nc^n}\\
&= \bar{a_0} + \bar{a_1}\bar{c} + \bar{a_2}\bar{c^2} + \ldots + \bar{a_n}\bar{c^n} \\
&= a_0 + a_1 \bar{c} + a_2\bar{c^2} + \ldots +a_n\bar{c^n}\\
&= f(\overline{c})
\end{align*}
\end{proof}
Using this lemma we can prove a very important result for polynomials in $\R[x]$

\begin{thm}{}{}
Let $f\in \R[x]$ be a non-constant polynomial. Then $f$ is irreducible in $\R[x]$ if and only if $\deg f = 1$ and $\deg f = 2$ 
\end{thm}
\begin{proof}
If $\deg f = 1$ or $\deg f = 2$ has no real roots then we know that $f$ is irreducible. Conversely if $f\in \R[x]$ is an irreducible polynomial then $f$ is also a polynomial in $\C[x]$. Since $\C$ is algebraically closed there is a $c\in \C$ such that $f(c) = 0$. If we have $c\in \R$ then since $f$ is irreducible it forces us to $\deg f = 1$.  
\\
If $c\notin \R$ then $c\in \C$ then by the conjugate roots theorem $\bar{c}$ is also a root. To we can write $f(x) = (x-c)(x-\bar{c})h(x)$ for $h\in \C[x]$. Then let $g = (x-c)(x-\overline{c})$ we have 
\begin{align*}
(x-c)(x-\overline{c}) &= x^2 - (c+\overline{c})x + c\overline{c} \\
&= x^2 - (2\Re c)x + |c|^2 \in \R[x]
\end{align*}
So $f$ is divisible by $g\in \R[x]$
\\
Thus we have a factorization $f = gh$ in $\R[x]$ where $\deg g = 2$. Again by irreducibility of $f$, $h$ must be a constant polynomial. Therefore $\deg	 f = 2$      
\end{proof}
\newpage
\subsection{Irreducible Polynomials in $\Q[x]$}

\begin{thm}{Rational Root Theorem}{}
Let $f\in \Q[x]$ be a non-constant polynomial and suppose $r\in \Q$ is a root of $f$, and let 
\[
f = a_0 + a_1x^1 + a_2x^2 + \ldots + a_nx^n 
\]
With $a_0, a_1, \ldots a_n \in \Z$ and $a_n\neq 0$ if we write $r = \frac{p}{q}$ with $\gcd (p, q) = 1$ (Simplest form). Then we must have 
\[
q\mid a_n \quad p \mid a_0 \text{ in $\Z$}
\]
\end{thm}
\begin{proof}
Since $\frac{p}{q}$ is a root we have 
\begin{align*}
a_0 + a_1 \frac{p}{q} + \ldots + a_n \left(\frac{p}{q} \right) = 0
\end{align*}
Multiplying by $q^n$ we have 
\begin{align*}
a_0q^n + a_1pq^{n-1} + \ldots + a_np = 0 \\
a_0q^n = -(a_1pq^{n-1} + \ldots + a_np) = -p(a_1q^{n-1} + \ldots a_n)
\end{align*}
This shows that we have $p\mid a_0q^n$. Since $\gcd(p,q) = 1$ we have $\gcd(p,q^n) = 1$ and we get $p\mid a_0$ by the divisibility results. 
\\
Similarly we can isolate $a_np^n$ to get $q\mid a_np^n$ and again using $\gcd(p^n, q) = 1$ we have $q\mid a_n$
\end{proof}

\begin{example}
Let $f(x)= x^3 -2x + 5\in \Q[x]$ is irreducible.  Since $\deg f = 3$ we know that it has no roots. We can also use the rational roots theorem. If $\frac{p}{q}$ is a root with $\gcd(p,q) = 1$ then we must have $q\mid 1$ and $p\mid 5$. This means we have $q\in \{-1,1\}$ and $p\in \{-1,1,-5,5\}$. So the possibilities for $\frac p q$ are $1,-1,-5,5$ and none of them are roots. 
\end{example}
\newpage
\subsection{Showing a number is irrational}
We can also use rational root theorem to show that some $\alpha \in \R$ is irrational. 
\begin{algo}{}{}
\begin{itemize}
\item[\#1] Find a non-zero polynomial for which $f$ with integer coefficients for which $\alpha$ is a root.  
\item[\#2] Use RRT to show that $f$ has no rational roots. 
\end{itemize}
\end{algo}
These 2 steps combined should prove any $\alpha \in \R$ is irrational. 

\begin{example}
$\alpha = \sqrt{2} + \sqrt{3}$ is irrational. We have $\alpha^2 = 2 + 2\sqrt{6} + 3 = 5+2\sqrt{5}$ so $\alpha^2 - 5 = 2\sqrt{6}$ and then $(\alpha^2 -5)^2 = (2\sqrt{6})^2 = 24$. So we have $\alpha^4 -10\alpha^2 +1 =0$. So all roots are integers and if $\alpha \in \Q$ then we must have $\alpha = \frac{p}{q}$ with $p\mid 1$ and $q\mid 1$ the only choices are $\frac{p}{q} = 1,-1$ and $f(1),f(-1)\neq 0$ so we cannot have $\alpha\in \Q$ 
\end{example}

\begin{example}
For any prime $p$ and $n\geq 2$ $\sqrt[n]{p}$ is irrational. 
\begin{align*}
\left(\sqrt[n]{p}\right)^n = p \\
\left(\sqrt[n]{p}\right)^n - p = 0
\end{align*}
So a polynomial with integer coefficients with root $\sqrt[n]{p}$ is $x^n - p$. Then if $\sqrt[n]{p}\in \Q$ then we have $\sqrt[n]{p} = \frac{r}{s}$. Then since it is a root of $x^n - p$ we must have $s\mid 1$ and $r\mid -p$ that means $s\in \{-1,1\}$ and $r\in \{-1,1,p,-p\}$ then the possibilities for $\frac{p}{q}$ are $1,-1,-p,p$. By simple computation we know that 
\begin{align*}
(-1)^n - p \neq 0 \\
(1)^n - p \neq 0 
\end{align*}
For the other possibility assume we have $p^n - p = 0$  then $p(p^{n-1}-1) = 0$ so we must have $p=0$ or $p^{n-1} - 1 =0$ the first case directly leads to a contradiction and in the second case if we have $p^{n-1} - 1 = 0$ that means $p^{n-1} = 1$ which is not true for any prime $p$ so we  have $\sqrt[n]{p}\notin \Q$ 
\end{example}




































































\end{document}
